%%%%%%%%%%%%%%%%%%%%%%%%%%%%%%%%%%%%%%%%%%%%%%%%%%%%%%%%%%%%%%%%%%%%%%%%%%%%%
%%%%%%                                                                  %%%%% 
%%%%%%          Maqueta de memòria TFC/PFC de l'EETAC                   %%%%% 
%%%%%%                                                                  %%%%% 
%%%%%%%%%%%%%%%%%%%%%%%%%%%%%%%%%%%%%%%%%%%%%%%%%%%%%%%%%%%%%%%%%%%%%%%%%%%%%
%%%%%%%%%%%%%%%%%%%%%%%%%%%%%%%%%%%%%%%%%%%%%%%%%%%%%%%%%%%%%%%%%%%%%%%%%%%%%
%%                                                                         %%
%%          Autor: Xavier Prats i Menéndez (xavier.prats@upc.edu)          %% 
%%                  Technical University of Catalonia (UPC)                %%
%%                                                                         %%
%%%%%%%%%%%%%%%%%%%%%%%%%%%%%%%%%%%%%%%%%%%%%%%%%%%%%%%%%%%%%%%%%%%%%%%%%%%%%
%%      This work is licensed under the Creative Commons  Attribution-     %%
%%   -Noncommercial-Share Alike 3.0 Spain License. To view a copy of this  %% 
%%    license, visit http://creativecommons.org/licenses/by-nc-sa/3.0/es/  %%
%%    or send a letter to Creative Commons, 171 Second Street, Suite 300,  %%
%%                  San Francisco,California, 94105, USA.                  %%
%%%%%%%%%%%%%%%%%%%%%%%%%%%%%%%%%%%%%%%%%%%%%%%%%%%%%%%%%%%%%%%%%%%%%%%%%%%%%
%% Versió 2.1 - Juliol 2012                                                %%
%%%%%%%%%%%%%%%%%%%%%%%%%%%%%%%%%%%%%%%%%%%%%%%%%%%%%%%%%%%%%%%%%%%%%%%%%%%%%

%%% NOTA: els seguents packages son necessaris per utilitzar la
%%%       plantilla seguent:
%%%       ifthen,calc,helvet,pslatex,fancyhdr,nextpage,subfigure,tocloft,graphicx,url

%%% NOTA: Es possible que algunes distribuicions Linux o Windows.
%%%       no portin aquests paquets instal·lats per defecte.
%%%       En aquest cas els haureu d'instal·lar manualment.


%%%%%%%%%%%%%%%%%%%%%%%%%%%%%%%%%%%%%%%%%%%%%%%%%%%%%%%%%%%%%%%%%%%%%%%%%%%%%
% 1- INICIALITZACIÓ
%%%%%%%%%%%%%%%%%%%%%%%%%%%%%%%%%%%%%%%%%%%%%%%%%%%%%%%%%%%%%%%%%%%%%%%%%%%%%

\documentclass[english,final]{setup/eetac_tfc_pfc}
%% * OPCIONS A CONFIGURAR al \documentclass
%%    - Estat del document: final o draft
%%      NOTA: Draft no inserta les figures i marca només l'espai que
%%      ocupen. També s'indica quan el text sobrepassa els marges.
%%      Draft és molt útil per compilar ràpid el document si no és important
%%      en aquell moment visualitzar les figures.
%%    - Idioma PRINCIPAL del document: catalan, spanish, english, french...

\usepackage[english]{babel}
%%  * INCLOURE TOTS ELS IDIOMES QUE S'USARAN EN EL DOCUMENT
%%    NOTA: per canviar d'idioma al mig del document usar:
%%          \selectlanguage{nom_idioma}
%%%%%%%%%%%%%%%%%%%%%%%%%%%%%%%%%%%%%%%%%%%%%%%%%%%%%%%%%%%%%%%%%%%%%%%%%%%%%

%%%%%%%%%%%%%%%%%%%%%%%%%%%%%%%%%%%%%%%%%%%%%%%%%%%%%%%%%%%%%%%%%%%%%%%%%%%%%
% 2- CÀRREGA DE PAQUETS ADICIONALS (OPCIONALS)
%%%%%%%%%%%%%%%%%%%%%%%%%%%%%%%%%%%%%%%%%%%%%%%%%%%%%%%%%%%%%%%%%%%%%%%%%%%%%

%%% NOTA: Es possible que algunes distribuicions Linux o Windows.
%%%       no portin aquests paquets instal·lats per defecte.
%%%       En aquest cas els haureu d'instal·lar manualment.

%% El paquet inputenc és extramadament útil. 
%% Permet escriure els accents directament amb l'editor de texte
%% sense haver de fer coses com per exemple: introducci\'o
%% Heu d'especificar la codificació de caracters que utilitzeu pel
%% vostre fitxer (en aquest exemple utf8)
\usepackage[utf8]{inputenc}

%% Símbols matemàtics de la American Mathematical Society
\usepackage{amssymb, amsmath, amsfonts}  

%% El paquet array proporciona eines molt útils a l'hora de fer 
%% equacions amb matrius
\usepackage{array}             

%% Paquet que permet fer taules fusionant cel·les de files consecutives
\usepackage{multirow}          

%% Paquet molt útil en cas de tenir taules molt llargues que 
%   ocupin vàries pàgines
\usepackage{longtable}          

%% Permet canviar els colors del document
\usepackage{color,colortbl}

%% Paquet molt útil que permet activar links en el PDF final.
\usepackage[
  	% CUSTOM PDF 
  	pdfauthor={Fernando Román García},
  	pdftitle={Master thesis - Fernando Román García},
	pdfsubject={An enhanced SleuthKit GUI for digital forensics},
	pdfkeywords={digital, forensics, sleuth, kit, GUI},
	%
  	pdfcreator={EETAC-UPC},
	pdfproducer={LaTeX, dvipdf},
	pdfdisplaydoctitle=true,
  	plainpages=false,
	linktocpage=true,
	colorlinks=true,
  	linkcolor=blue,
	citecolor=blue,
	urlcolor=blue,
	hyperfootnotes=false,
  	pagebackref=true,
	pdfpagelabels=true,
	pdfpagemode=UseOutlines,
]{hyperref} 

%% NOTA IMPORTANT!:
%% Per tal que hyperef funcioni correctament amb els capitols o seccions no
%% numerats (\chapter*{}), com per exemple introducció, conclusions i
%% bibliografia cal posar les dues comandes seguents ABANS del \chapter*{} en
%% questió
%\cleardoublepage \phantomsection

%% Permet trencar links URL. 
%% Atenció! afegir aquest paquet DESPRES del hyperref!!
\usepackage{breakurl} 

%% Permet arranjar matricialment multiples figures
%% NOTA: afegir aquest paquet DESPRES del hyperref!!
%%       Si no es desitja utilitzar aquest paquet, comentar la linia seguent
%%       i anar TAMBE al fitxer de classe (eetac_tfc_pfc.cls) per substituir: 
%%       \RequirePackage[subfigure]{tocloft}  per  \RequirePackage{tocloft}
\usepackage{subfigmat}

% Custom packages
\usepackage{xparse}
\usepackage{listings}
\usepackage{rotating}
\usepackage{pdflscape}
\usepackage{wrapfig}
\usepackage{etoolbox}

%%%%%%%%%%%%%%%%%%%%%%%%%%%%%%%%%%%%%%%%%%%%%%%%%%%%%%%%%%%%%%%%%%%%%%%%%%%%%
% 3- DOCUMENT
%%%%%%%%%%%%%%%%%%%%%%%%%%%%%%%%%%%%%%%%%%%%%%%%%%%%%%%%%%%%%%%%%%%%%%%%%%%%%

%%% Configuració de les dades i variables boleanes rellevants del document:
\input{setup/dades.tex}  

%%% Configuració de MACROS o ENTORNS (opcionals) definides per l'usuari:
\input{setup/user-macros.tex}  

%%% Configuració manual de les regles d'hyphenation:
\input{setup/hyphenation.tex}  

%%% Configurations of programming languages highlights 
%%% MOVE THIS TO MACRO FILE?? %%%

% Colors %%%%%%%%%%%%%%%%%%%%%%%%%%%%%%%%%%%%%%%%%%%%%%%%%%%%%%%%%%%%%%%%%%%%%%%

\definecolor{dkgreen}{rgb}{0,0.6,0}
\definecolor{gray}{rgb}{0.5,0.5,0.5}
\definecolor{mauve}{rgb}{0.58,0,0.82}
\definecolor{deepblue}{rgb}{0,0,0.5}
\definecolor{deepred}{rgb}{0.6,0,0}
\definecolor{deepgreen}{rgb}{0,0.5,0}

\definecolor{pictonblue}{RGB}{62, 193, 233}
\definecolor{turquoiseblue}{rgb}{0.388,0.831,0.960}
\definecolor{saffron}{rgb}{0.372,0.211,0.988}
\definecolor{rawsienna}{rgb}{0.819,0.576,0.255}

\definecolor{brightedred}{RGB}{161,0,0}
\definecolor{gigas}{RGB}{72, 62, 170}


% Custom macros %%%%%%%%%%%%%%%%%%%%%%%%%%%%%%%%%%%%%%%%%%%%%%%%%%%%%%%%%%%%%%%%

\newcommand{\lstcustomdefaults}{
	\setkeys{lst}{
		basicstyle=\footnotesize\ttfamily,
		aboveskip=3mm,
		belowskip=3mm,
		showstringspaces=false,
		columns=flexible,
		numbers=left,
		numberstyle=\footnotesize,
		breaklines=true,
		breakatwhitespace=true,
		tabsize=3,
	}
}
\newcommand{\newproglanguage}[2]{
	\expandafter\newcommand\csname #1style\endcsname[1][]{
		#2
		\lstcustomdefaults
		\setkeys{lst}{##1}
	}
	
	\lstnewenvironment{#1}[1][]
	{
		\expandafter\csname #1style\endcsname
		\lstset{##1}
	}
	{}
	
	\expandafter\DeclareDocumentCommand\csname #1external\endcsname
	{ O{} m }
	{
		\expandafter\csname #1style\endcsname
		\lstinputlisting[##1]{##2}
	}
	
	\expandafter\newcommand\csname #1inline\endcsname
	[1]{{\expandafter\csname #1style\endcsname\lstinline!##1!}}
}

%% Code figure
\newenvironment{codefigure}[2]{
	\begin{figure}[htb]
		\begin{quote}
			\def\codeFigCaption{#1}
			\def\codeFigLabel{#2}
}
{
			\caption{\codeFigCaption}
			\label{\codeFigLabel}
		\end{quote}
	\end{figure}
}

%% Bash code
\newenvironment{bashcode}{
	\begin{center}\itshape\bfseries
}
{
	\end{center}
}


% Style highlighting %%%%%%%%%%%%%%%%%%%%%%%%%%%%%%%%%%%%%%%%%%%%%%%%%%%%%%%%%%%

%% Java
\newproglanguage{java}{
	\lstset{
		language=Java,
		basicstyle={\small\ttfamily},
		numberstyle=\tiny\color{gray},
		keywordstyle=\color{blue},
		commentstyle=\color{dkgreen},
		stringstyle=\color{mauve},
	}
}

%% Python
\newproglanguage{python}{
	\lstset{
		language=Python,
		otherkeywords={self},             	% Add keywords here
		keywordstyle=\color{deepblue},
		emph={
			__init__,
			TestApp,
			App,
			Button
		},          				    	% Custom highlighting
		emphstyle=\color{deepred},    		% Custom highlighting style
		stringstyle=\color{deepgreen},
	}
}

%% C++
\newproglanguage{cpp}{
	\lstset{
		language=C++,
		% Language settings
		commentstyle=\color{dkgreen}\ttfamily,
		identifierstyle=\color{black},
		stringstyle=\color{red}\ttfamily,
		% Syntax keywords
		classoffset=0,
		keywordstyle=\color{blue},
		% Classes
		classoffset=1,
		keywordstyle=\color{turquoiseblue}\bfseries,
	}
}


%% JavaScript
\lstdefinelanguage{JavaScript}{
	% Language settings
	sensitive=true,
	% Keywords
	classoffset=0,
	keywords={typeof, new, true, false, catch, function, return,
		null, catch, switch, var, if, in, while, do, else, case,
		break, const, let, class, export, boolean, throw, implements,
		default, import, this, extends},
	% Classes
	classoffset=1,
	keywords={Math},
	% Comments
	comment=[l]{//},
	morecomment=[s]{/*}{*/},
	% Strings
	morestring=[b]',
	morestring=[b]",
}
\newproglanguage{js}{
	\lstset{
		language=JavaScript,
		% General theme
		commentstyle=\color{dkgreen}\ttfamily,
		identifierstyle=\color{black},
		stringstyle=\color{red}\ttfamily,
		% Syntax keywords
		classoffset=0,
		keywordstyle=\color{gigas}\bfseries,
		% Classes
		classoffset=1,
		keywordstyle=\color{turquoiseblue}\bfseries,
	}
}


%% JSX
\lstdefinelanguage{JSX}{
	% Language settings
	alsoletter={<, >, \/},
	sensitive=true,
	% Keywords
	classoffset=0,
	keywords={typeof, new, true, false, catch, function, return,
		null, catch, switch, var, if, in, while, do, else, case,
		break, const, let, class, export, boolean, throw, implements,
		default, import, this, extends},
	% HTML
	classoffset=2,
	keywords={<div>, <\/div>, />},
	% Comments
	comment=[l]{//},
	morecomment=[s]{/*}{*/},
	% Strings
	morestring=[b]',
	morestring=[b]",
}
\newproglanguage{jsx}{
	\lstset{
		language=JSX,
		% General theme
		commentstyle=\color{dkgreen}\ttfamily,
		identifierstyle=\color{black},
		stringstyle=\color{red}\ttfamily,
		% Syntax keywords
		classoffset=0,
		keywordstyle=\color{gigas}\bfseries,
		% Classes
		classoffset=1,
		keywordstyle=\color{turquoiseblue}\bfseries,
		% HTML
		classoffset=2,
		keywordstyle=\color{black}\bfseries,
	}
}

%% Haxe
\lstdefinelanguage{Haxe}{
	% General theme
	sensitive=true,
	% Syntax keywords
	classoffset=0,
	keywords={class, static, public, function, var},
	% Classes
	classoffset=1,
	keywordstyle={String, Void},
	% Comments
	comment=[l]{//},
	morecomment=[s]{/*}{*/},
	% Strings
	morestring=[b]',
	morestring=[b]"
}
\newproglanguage{haxe}{
	\lstset{
		language=Haxe,
		% General theme
		commentstyle=\color{dkgreen}\ttfamily,
		stringstyle=\color{red}\ttfamily,
		identifierstyle=\color{black},
		% Syntax keywords
		classoffset=0,
		keywordstyle=\color{blue}\bfseries,
		% Classes
		classoffset=1,
		keywordstyle=\color{turquoiseblue}\bfseries,
	}
}


%% Sass
\lstdefinelanguage{Sass}{
	alsoletter={-, :, \#, .},
	sensitive=false,
	% CSS properties
	classoffset=0,
	keywords={background-color:, color:, display:, flex:},
	% SCSS @ methods
	classoffset=1,
	keywords={@include, @mixin},
	% User defined variables and classes
	classoffset=2,
	% Comments
	comment=[l]{//},
	morecomment=[s]{/*}{*/},
	% Strings
	morestring=[b]',
	morestring=[b]"
}
\newproglanguage{sass}{
	\lstset{
		language=Sass,
		% General theme
		commentstyle=\color{dkgreen}\ttfamily,
		stringstyle=\color{red}\ttfamily,
		identifierstyle={\color{gigas}\small\ttfamily},
		% CSS properties
		classoffset=0,
		keywordstyle=\color{brightedred}\bfseries,
		% SCSS @ methods
		classoffset=1,
		keywordstyle=\color{black}\bfseries,
		% User defined variables and classes
		classoffset=2,
		keywordstyle=\color{pictonblue},
	}
}



\begin{document}

%% Seleccionar l'idioma principal del document:
\selectlanguage{english}

\beforepreface  

%% RESUM i OVERVIEW
%%%%%%%%%%%%%%%%%%%%%%%%%%%%%%%%%%%%%%%%%%%%%%%%%%%%%%%%%%%%%%%%%%%%%%%%%%%%%
% NOTA: les longituds passades com a parametres d'entrada  s'han d'ajustar
% manualment fins que el requadre del resum/overview ocupi tota la pàgina. 

%%% Resum en català (o castellà)
%\begin{resum}{10cm} Aquest document conté les pautes del format de presentació
%del treball o projecte de %final de carrera. En tot cas, cal tenir en compte
%el que estableix la ``Normativa del %treball de fi de carrera (TFC) i del
%projecte de fi de carrera (PFC)'' aprovada per la %Comissió Permanent de
%l'EETAC, especialment l'apartat ``Requeriments del treball''.  \end{resum}

%%% Resum en anglès
\begin{overview}{11cm} This document contains guidelines for writing your
TFC/PFC. However, you should also take into consideration the standards
established in the document Normativa del treball de fi de carrera (TFC) i del
projecte de fi de carrera (PFC), paying special attention to the section
Requeriments del treball, as this document has been approved by the EETAC
Standing Committee \end{overview}


%NOTA: En cas d'utilitzar l'espanyol com a idioma principal del document, el
%latex anomena les taules com a 'Cuadros'. Si es desitja canviar aquesta
%nomenclatura i utilitzar la paraula 'Tabla' descomentar les línies següents:
%\def\listtablename{Índice de tablas} \def\tablename{Tabla}%



% Amb aqueta comanda indiquem que ja s'han inclòs tots els apartats del prefaci
% del projecte o podem començar a incloure els capitols de la memòria
\afterpreface


%%%%%%%%%%%%%%%%%%%%%%%%%%%%%%%%%%%%%%%%%%%%%%%%%%%%%%%%%%%%%%%%%%%%%%%%%%
%%%%%% INCLOURE A PARTIR D'AQUÍ TOTS ELS CAPÍTOLS DE LA MEMORIA   %%%%%%%%
%%%%%%%%%%%%%%%%%%%%%%%%%%%%%%%%%%%%%%%%%%%%%%%%%%%%%%%%%%%%%%%%%%%%%%%%%%

% NOTA: recordar que la introducció i les conclusions són capítols NO
% enumerats, per tant s'ha d'usar \chapter*

% NOTA: és aconsellable incloure els capítols de la memòria en fitxers separats
% utlitzant la comanda \input  Per exemple: \input{capitol1}  que farà que
% s'inclogui el fitxer capitol1.tex

% NOTA: Si es vol incloure agraïments i/o glosari, fer-ho utilitzant
% \chapter*{} i incloure'ls abans la introducció

\cleardoublepage
\phantomsection
\chapter*{Introduction}

% TODO: REVISAR
Nowadays it's impossible to think in live without digital technologies. They
are involved in most of our daily actions, such as, to talk with friends, to
read news and entertainment. For this reason, our electronic devices can be
considered as digital sources of information that can be used to prove even
what it's owner has done at a specific moment.

Digital forensics science, or digital forensics, is defined as:

\begin{quote}
"The use of scientifically derived and proven methods toward the preservation,
collection, validation, identification, analysis, interpretation,
documentation, and presentation of digital evidence derived from digital
sources for the purpose of facilitation or furthering the reconstruction of
events found to be criminal, or helping to anticipate unauthorized actions
shown to be disruptive to planned operations" \cite{DFRWS-df-road-map}
\end{quote}

In several cases, when a search warrant is given to investigators, they can
proceed with the acquisition and analysis of a suspect's digital data source,
for instance, hard disks (HDD). Using this information, they can look for
evidences that prove the suspect's innocence or guiltiness.

There are many ways to do this analysis but the more generic one presented in
\cite{ds-phases} consists in:

\begin{description}
	\item [Pre-Process]
		Even sometimes this step could be forgotten, is one of the more
		important. It consists in all the works needed to be done before
		the collection of data. To have all the approvals for relevant
		authorities is considered here.

	\item [Acquisition \& Preservation]
		Once everything is prepared, now investigator can acquire the
		data. An image and a hash of the image have to be computed. The
		image is an exact copy of the data and the hash is a small piece
		of information that summarizes the image.

		That way, if anyone modifies the image and the hash is computed
		again, a different value will be retrieved. So this procedure
		stands to guarantee the proof integrity.

	\item [Analysis]
		Is the main phase of the investigation. It consists on extract
		the relevant information acquired before. Many different 
		techniques can be used: data recovery, words lookup, etc. Note
		that, on this step, many different tools may be used to be able
		to perform a proper analysis.

	\item [Presentation]
		A good analysis is useless if it's not presented in a easy to 
		understand manner. After this phase, the innocence of the
		suspect has to be proved or refused.

	\item [Post-Process]
		Physical (i.e. HDD) and digital proofs have to be returned and
		a document must be written to finish the case.
\end{description}

There are many digital forensics analysis tools but most of them are not
supported on all the operative systems. A very complete software is 
\textbf{EnCase Forensic} \cite{encase-web}. It has a lot of tools that can be 
used from the acquisition step until the final report. But it's main 
disadvantage is that is not open source.

Another well known software regarding digital forensics is \textbf{The Sleuth
Kit} \cite{tsk-web}. This open source cross-platform toolkit is a collection 
of terminal applications and a C library that allows you to analyze disk images 
and recover files from them. The Sleuth Kit team also developed
\textbf{Autopsy}, with is a windows user interface to efficiently analyze hard 
drives and smart phones.

Due to the lack of open source cross-platform tools with user interface, the
objective of this project develop one with main functionalities. The start point
is to use \textbf{The Sleuth Kit}'s C library since it's cross-platform and a 
very mature project.


% \chapter{Digital forensics}

% TODO
De wikipedia! Redatar otra vez

Digital forensics (sometimes known as digital forensic science) is a branch of forensic 
science encompassing the recovery and investigation of material found in digital devices,
often in relation to computer crime.[1][2] The term digital forensics was originally used
is a synonym for computer forensics but has expanded to cover investigation of all devices 
capable of storing digital data.[1] With roots in the personal computing revolution of
the late 1970s and early 1980s, the discipline evolved in a haphazard manner during the 
1990s, and it was not until the early 21st century that national policies emerged.

The goal of computer forensics is to explain the current state of a digital artifact; such
as a computer system, storage medium or electronic document.[38] The discipline usually
covers computers, embedded systems (digital devices with rudimentary computing power and
onboard memory) and static memory (such as USB pen drives).

Computer forensics can deal with a broad range of information; from logs (such as internet
history) through to the actual files on the drive.

In 2007 prosecutors used a spreadsheet recovered from the computer of Joseph E. Duncan III
to show premeditation and secure the death penalty.[3] Sharon Lopatka's killer was
identified in 2006 after email messages from him detailing torture and death fantasies were
found on her computer.

\section{Phases}

% TODO
SOME TEXT HERE

\begin{description}
	\item [Attribution]
		Meta data and other logs can be used to attribute actions to an individual. 
		
		For example, personal documents on a computer drive might identify its owner.
	\item [Alibis and statements]
		Information provided by those involved can be cross checked with digital 
		evidence.
		
		For example, during the investigation into the Soham murders the offender's
		alibi was disproved when mobile phone records of the person he 	claimed to be
		with showed she was out of town at the time.
	\item [Intent]
		As well as finding objective evidence of a crime being committed, investigations
		can also be used to prove the intent (known by the legal term mens rea).
		
		For example, the Internet history of convicted killer Neil Entwistle included 
		references to a site discussing How to kill people.
	\item [Evaluation of source]
		File artifacts and meta-data can be used to identify the origin of a particular
		piece of data; for example, older versions of Microsoft Word embedded a Global
		Unique Identifer into files which identified the computer it had been created on. 
		Proving whether a file was produced on the digital device being examined or
		obtained from elsewhere (e.g., the Internet) can be very important.[3]
	\item [Document authentication]
		Related to "Evaluation of source," meta data associated with digital documents
		can be easily modified (for example, by changing the computer clock you can affect
		the creation date of a file). Document authentication relates to detecting and
		identifying falsification of such details.
\end{description}

\section{Tools}

% TODO
What's a digital forensic tool??

\subsection{The Sleuth Kit and Autopsy}



\chapter{Cross-platform development}

At the beginning of computers, each hardware manufacturer had to decide how to
access to its hardware functionalities. Since not all of them were decided to be
the same the fist inter-compatibility problems started to appear because 
programs were able to run on different computers.

To reduce this problems drivers were created. A driver is a piece of code that 
maps logical functions (for instance, turn on a led) to hardware operations 
(short circuit some part of the hardware). Logic functions are provided to 
programs by the Operative Systems (OS) so the problem now is that not all of 
them can run on all the OS.

\section{Technological soup}

Nowadays lots of technologies has been created with the goal of writing just
one code to run in many platforms as possible. This reduces a lot the
development cost by reducing performance or other factors .

The most common way to achieve this goal is with scripting or virtual machine 
programming languages. Those languages are interpreted on execution time. 
Therefore programs created with this kind of technologies can run on all
platforms supported by the interpreter.

Another kind of solutions are transpilers or source-to-source compilers. They 
translate from one programming language to many others. This languages are
able to be executed on many platforms.

\subsection{Java}

This well known programming language is one of the most used to build
cross-platform applications due to its maturity. 

\subsection{Qt}

\subsection{Python}

Using Kivy, a python GUI framework.

\subsection{Electron}

\subsection{NW.js}

\subsection{Haxe}

Native programming language

\section{Electron}

\section{Web technologies}

\subsection{Typescript}

\subsection{Sass}

\subsection{Gulp}


\chapter{Software architecture}

% TODO

\section{React}

\section{Redux}

\section{React + Redux integration}

\section{Complex-behaviours: redux-observables}

\subsection{rxjs}

\section{TheSleutKit javascript wrapper}


\chapter{The Sleuth Kit JavaScript}

The Sleuth Kit (TSK) is an open source cross-platform collection of terminal
applications and a C library that allows investigators to analyze disk images
and recover files from them \cite{tsk-web}. It is divided in file system and
volume tools. 

The fist group, file system tools, serve to examine the file system of a suspect
computer in a non-intrusive way. Main file system tools are shown on Figure
\ref{T:tsk-fs-tools}.

\begin{table}[htb]
\begin{center}
\begin{tabular}{|l|l|l|}
\hline
{\bf Command }	& {\bf Description }  \\ \hline \hline
fsstat & Shows file system details and statistics \\ \hline \hline
fls	& Lists allocated and deleted file names in a directory \\ \hline
ffind & Finds allocated and unallocated file names \\ \hline \hline
icat & Extracts the data units of a file \\ \hline
\end{tabular}
\caption{Main file system tools from TSK \cite{tsk-tools-wiki}}
\label{T:tsk-fs-tools}
\end{center}
\end{table}

Volume tools are in charge of the layout of disks and another media. They are
needed to identify where partitions are located to be able to proceed with a
file system analysis. Table \ref{T:tsk-v-tools} lists main volume tools.

\begin{table}[htb]
\begin{center}
\begin{tabular}{|l|l|l|}
\hline
{\bf Command }	& {\bf Description }  \\ \hline \hline
mmls & Displays the layout of a disk \\ \hline \hline
mmstat & Display details about a volume system \\ \hline
\end{tabular}
\caption{Main volume tools from TSK \cite{tsk-tools-wiki}}
\label{T:tsk-v-tools}
\end{center}
\end{table}


\chapter{Img-spy}

\section{FOO}

\subsection{Explorer}

\subsection{Timeline}

\subsection{Search}

\section{Code quality analysis?}

\cleardoublepage
\phantomsection
\chapter*{Conclusions}

The development of a cross-platform digital forensics applications is a long
process with many key decisions. The first one is to choose a technological
environment. This task is not that easy due to the great amount of very
different solutions. But, nowadays web technologies are standing strong in
terms of cross-platform compatibility. \textbf{A well-known solution} that many
companies are using today is \textbf{Electron} \cite{electron-web}, which uses
Chromium as library to instantiate Chrome windows. 

The environment to work is well-defined so now the overview of the application
have to be decided. The software architecture defines it by using black boxes
and defining their interactions. Model-view-controller (MVC) is a mature
widely-used architecture that fits very well with applications that have a
little amount of views with interaction between themselves. This is not the
case because digital forensic tools proposed have those interactions.

The direct evolution of MVC is Flux, with was defined by Facebook to solve this
problem, and to build a \textbf{flux-like architecture, React-Redux} works
pretty well. But to complement their lacks, some other libraries, such as
Redux-Observables, were used.

With those good bases, a complete application can be build. But it has more to 
go before start the application development because there are many
computationally complex operations. Those have to be coded in a low programming
language to guarantee a good efficiency. The Sleuth Kit provides a C library
that implements those operations. Therefore, \textbf{The Sleuth Kit JavaScript,
a Node.js wrapper, has been created} to let JavaScript use those operations.

Finally, \textbf{Img-Spy, the digital forensics application developed to fulfill
the goal of this project, defines tree tools}: \textit{Explorer},
\textit{Timeline} and \textit{Search}. Those tools let an investigator analyze
the file system of an image in a non-intrusive way, create a timeline based on
actions performed on the disk and search which files contain a specific string.

During the development process, many interesting or difficult tasks where found,
for instance, to watch changes on the file system of user's computer, perform
actions in parallel using JavaScript and define a friendly user interface
supporting, for example, resize-panels or multiple themes.

Looking ahead, many optimizations and features can be added to Img-spy. Fist,
\textit{Explorer} tool can be improved in terms of performance adding Redux
selectors and reducing React rerenders. Also, the hash is always computed
without asking the user consuming, too many resources and not always is needed.

\textit{Timeline} tool also can be improved adding graphs to represent those
actions in more visual way. Add filters can also help investigators to remove
useless information. For instance, a useful filter could be to see just files
of a specific search result.

In many cases, search files by name is also needed. Then, \textit{Search} tool
can implement this operation. More custom searches can be executed using Regex.

There is a saying that “Programming can be fun, so can cryptography; however
they should not be combined” \cite{code-complete}. So regarding code quality,
comments can be added to help new developers understand easily how the software 
works.

Finally, new tools can be added in order to detect mismatches on file 
extensions or to help investigators to write the final report. But always
taking into account, as Gordon Bell said:

\begin{quote}
	“Every big computing disaster has come from taking too many ideas and putting them in one place”.
\end{quote}

%%%  BIBLIOGRAFIA
%%%%%%%%%%%%%%%%%%%%%%%%%%%%%%%%%%%%%%%%%%%%%%%%%%%%%%%%%%%%%%%%%%%%%%%%%%

\input{bibliography/bibliography}

%%%%%%%%%%%%%%%%%%%%%%%%%%%%%%%%%%%%%%%%%%%%%%%%%%%%%%%%%%%%%%%%%%%%%%%%%%
%%%%%%                           APENDIXS                         %%%%%%%%
%%%%%%%%%%%%%%%%%%%%%%%%%%%%%%%%%%%%%%%%%%%%%%%%%%%%%%%%%%%%%%%%%%%%%%%%%%
\pagestyle{empty}  % no tocar

%% Descomentar una de les dues línies següents, en funció de:
%%  a) els apendixs s'encuadernaran apart (amb portada) 
%%  b) els apendixs s'enquadernen amb el mateix projecte (sense portada). 
%% Recordeu que si tot el document (amb apèndixs) excedeix les 100 pagines 
%% s'ha d'enquadernar a part
%\appendix\ambportada
\appendix\senseportada


%%%%%%%%%%%%%%%%%%%%%%%%%%%%%%%%%%%%%%%%%%%%%%%%%%%%%%%%%%%%%%%%%%%%%%%%%%
%%%%%% INCLOURE A PARTIR D'AQUI TOTS ELS CAPÍTOLS DELS APENDIXS   %%%%%%%%
%%%%%%%%%%%%%%%%%%%%%%%%%%%%%%%%%%%%%%%%%%%%%%%%%%%%%%%%%%%%%%%%%%%%%%%%%%

\input{appendix/electron-main}
\input{appendix/tsk-js-tpyings}
% \chapter{Definitions}

\begin{description}
	\item [Open source software]
	Very well Manuel

	\item [Cross-platform software]
	Very well Fandango

	\item [Operative System (OS)]
	Very well Manuel


\end{description}


%%%%%%%%%%%%%%%%%%%%%%%%%%%%%%%%%%%%%%%%%%%%%%%%%%%%%%%%%%%%%%%%%%%%%%%%%%
%%%%%%%%%%%%%%%%%%%%%%%%%%%%%%%%%%%%%%%%%%%%%%%%%%%%%%%%%%%%%%%%%%%%%%%%%%
%%%%%%%%%%%%%%%%%%%%%%%%%%%%%%%%%%%%%%%%%%%%%%%%%%%%%%%%%%%%%%%%%%%%%%%%%%
% i  aixo es tot! ;)
\end{document}






