\cleardoublepage
\phantomsection
\chapter*{Conclusions}

The development of a cross-platform digital forensics applications is a long
process with many key decisions. The first one is to choose a technological
environment. This task is not that easy due to the great amount of very
different solutions. But, nowadays web technologies are standing strong in
terms of cross-platform compatibility. \textbf{A well-known solution} that many
companies are using today is \textbf{Electron} \cite{electron-web}, which uses
Chromium as library to instantiate Chrome windows. 

The environment to work is well-defined so now the overview of the application
have to be decided. The software architecture defines it by using black boxes
and defining their interactions. Model-view-controller (MVC) is a mature
widely-used architecture that fits very well with applications that have a
little amount of views with interaction between themselves. This is not the
case because digital forensic tools proposed have those interactions.

The direct evolution of MVC is Flux, with was defined by Facebook to solve this
problem, and to build a \textbf{flux-like architecture, React-Redux} works
pretty well. But to complement their lacks, some other libraries, such as
Redux-Observables, were used.

With those good bases, a complete application can be build. But it has more to 
go before start the application development because there are many
computationally complex operations. Those have to be coded in a low programming
language to guarantee a good efficiency. The Sleuth Kit provides a C library
that implements those operations. Therefore, \textbf{The Sleuth Kit JavaScript,
a Node.js wrapper, has been created} to let JavaScript use those operations.

Finally, \textbf{Img-Spy, the digital forensics application developed to fulfill
the goal of this project, defines tree tools}: \textit{Explorer},
\textit{Timeline} and \textit{Search}. Those tools let an investigator analyze
the file system of an image in a non-intrusive way, create a timeline based on
actions performed on the disk and search which files contain a specific string.

During the development process, many interesting or difficult tasks where found,
for instance, to watch changes on the file system of user's computer, perform
actions in parallel using JavaScript and define a friendly user interface
supporting, for example, resize-panels or multiple themes.

Looking ahead, many optimizations and features can be added to Img-spy. Fist,
\textit{Explorer} tool can be improved in terms of performance adding Redux
selectors and reducing React rerenders. Also, the hash is always computed
without asking the user consuming, too many resources and not always is needed.

\textit{Timeline} tool also can be improved adding graphs to represent those
actions in more visual way. Add filters can also help investigators to remove
useless information. For instance, a useful filter could be to see just files
of a specific search result.

In many cases, search files by name is also needed. Then, \textit{Search} tool
can implement this operation. More custom searches can be executed using Regex.

There is a saying that “Programming can be fun, so can cryptography; however
they should not be combined” \cite{code-complete}. So regarding code quality,
comments can be added to help new developers understand easily how the software 
works.

Finally, new tools can be added in order to detect mismatches on file 
extensions or to help investigators to write the final report. But always
taking into account, as Gordon Bell said:

\begin{quote}
	“Every big computing disaster has come from taking too many ideas and putting them in one place”.
\end{quote}