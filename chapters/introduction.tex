\cleardoublepage
\phantomsection
\chapter*{Introduction}

Nowadays it is impossible to think our daily lifes without digital
technologies. They are involved in most of our daily actions, such as, to talk
with friends, to read news, documents transport and entertainment. For this
reason, our electronic devices can be considered as digital sources of 
information that can even prove possible alibis.

Digital forensics science, or digital forensics, is defined as:

\begin{quote}
“The use of scientifically derived and proven methods toward the preservation,
collection, validation, identification, analysis, interpretation,
documentation, and presentation of digital evidence derived from digital
sources for the purpose of facilitation or furthering the reconstruction of
events found to be criminal, or helping to anticipate unauthorized actions
shown to be disruptive to planned operations” \cite{DFRWS-df-road-map}
\end{quote}

In several cases, when a search warrant is given to investigators, they can
proceed with the acquisition and analysis of a suspect's digital data source,
for instance, hard disks (HDD). Using this information, they can look for
evidences that prove the suspect's innocence or guiltiness.

There are many ways to do this analysis, but the more generic one presented in
\cite{ds-phases}, consists on:

\begin{description}
	\item [Pre-Process]
		This is one of the most important steps; although it is sometimes 
		forgotten. It consists of all the tasks needed to be performed
		before the collection of data, including the necessary approvals 
		by relevant authorities.

	\item [Acquisition \& Preservation]
		Once everything is prepared, now the investigator can acquire 
		the data. An image and a digest of the analyzed data have to be
		computed. The image is an exact copy of the data and the digest is the
		result of a fixed-size unidirectional funcion, called hash, 
		that summarizes the image. Among other properties, a hash funtion must
		be collusion-resistant; that is to say, that it has to be
		computationally infeasible to find two images with the same digest or
		to modify a image in such a way the digest remains the same.

		That way, if anyone modifies the image and the digest is computed
		again using the modified image, a different value will be 
		retrieved. So this procedure stands to guarantee the proof 
		integrity.
		
		Although not very formal, in the rest of the document we will use the
		term hash to refer to both the hash function and the digest (the result
		of the hash function), for it is the most widely way of using it.

	\item [Analysis]
		This is the main phase of the investigation. It consists on extracting
		the relevant information acquired before. Many techniques can be 
		used: data recovery, words lookup, etc. Note that, on this step,
		some tools may be used to be able to perform a proper analysis.

	\item [Presentation]
		A good analysis is useless if it is not presented in an easy to 
		understand manner. After this phase, the innocence of the
		suspect has to be proven or refused.

	\item [Post-Process]
		Physical and digital proofs or evidences have to be returned and a
		report must be written to finish the case.

\end{description}

There are many digital forensics analysis tools. A very complete software is 
\textbf{EnCase Forensic} \cite{encase-web}. It has a lot of tools that can be 
used from the acquisition step until the final report. But its main disadvantage
is that it is not open source.

Another well-known software regarding digital forensics is \textbf{The Sleuth
Kit} \cite{tsk-web}. This open-source cross-platform toolkit is a collection 
of terminal applications and a C library that allows you to analyze disk images
and recover files from them. The Sleuth Kit team also developed
\textbf{Autopsy}, which is a Windows' user interface to efficiently analyze
hard drives and smart phones.

Due to the lack of open-source cross-platform tools with a friendly user
interface, the objective of this project is to develop one that eases the
forensic analysts' task addressing from the very beginning the main
functionalities of a digital forensics software. The start point is to use The
Sleuth Kit's C library  \cite{tsk-web} since it is a cross-platform very-mature
project.

In order to achieve this goal, \textit{Cross-platform development} (Chapter
\ref{S:cp-develompent}) will explain \textbf{why not all applications are
cross-platforms} and will show several options to build one. Finally, will talk
a little more in detail about the one that fits better with this project,
Electron \cite{electron-web}.

After the technological environment is decided, \textit{Software architecture}
(Chapter \ref{S:architecture}) talks about the evolution of 
model-view-controller to flux architecture and how it is implemented \textbf{to 
create a scalable application}.

\textit{Digital forensics analysis} (Chapter \ref{S:df-analysis}) get deeper in
how a digital forensics analysis can be done using The Sleuth Kit toolkit. Then,
it explains the development of an important part of this project, \textbf{The
Sleuth Kit JavaScript} (TSK-js), with a Node.js \cite{nodejs-web} C wrapper
that gives access to The Sleuth Kit functionalities on Node.js.

Finally, \textbf{\textit{Img-spy}} (Chapter \ref{S:img-spy}) is the name of
\textbf{the application that fulfills this project objective}. It starts with a
tour using the application and ends explaining some difficult or interesting
parts of the development.

