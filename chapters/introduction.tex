\cleardoublepage
\phantomsection
\chapter*{Introduction}

% TODO: REVISAR
Nowadays it's impossible to think in live without digital technologies. They
are involved in most of our daily actions, such as, to talk with friends, to
read news and entertainment. For this reason, our electronic devices can be
considered as digital sources of information that can be used to prove even
what it's owner has done at a specific moment.

Digital forensics science, or digital forensics, is defined as:

\begin{quote}
"the use of scientifically derived and proven methods toward the preservation,
collection, validation, identification, analysis, interpretation,
documentation, and presentation of digital evidence derived from digital
sources for the purpose of facilitation or furthering the reconstruction of
events found to be criminal, or helping to anticipate unauthorized actions
shown to be disruptive to planned operations" \cite{DFRWS-df-road-map}
\end{quote}

In several cases, when a search warrant is given to investigators, they can 
proceed with the analysis of a suspect's digital data source, mainly hard disks
(HDD).

This procedure consists, as described before on:

\begin{description}
	\item [Attribution]
		Meta data and other logs can be used to attribute actions to an
		individual. 
		
		For example, personal documents on a computer drive might
		identify its owner.
	\item [Alibis and statements]
		Information provided by those involved can be cross checked
		with digital evidence.
		
		For example, during the investigation into the Soham murders
		the offender's alibi was disproved when mobile phone records of
		the person he 	claimed to be with showed she was out of town
		at the time.
	\item [Intent]
		As well as finding objective evidence of a crime being
		committed, investigations can also be used to prove the intent
		(known by the legal term mens rea).
		
		For example, the Internet history of convicted killer Neil
		Entwistle included references to a site discussing How to kill
		people.
	\item [Evaluation of source]
		File artifacts and meta-data can be used to identify the origin
		of a particular piece of data; for example, older versions of
		Microsoft Word embedded a Global Unique Identifer into files
		which identified the computer it had been created on.  Proving
		whether a file was produced on the digital device being
		examined or obtained from elsewhere (e.g., the Internet) can be
		very important.[3]
	\item [Document authentication]
		Related to "Evaluation of source," meta data associated with
		digital documents can be easily modified (for example, by
		changing the computer clock you can affect the creation date of
		a file). Document authentication relates to detecting and
		identifying falsification of such details.
\end{description}

The objective of this document is to explain 

