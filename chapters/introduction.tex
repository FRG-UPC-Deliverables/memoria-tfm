\cleardoublepage
\phantomsection
\chapter*{Introduction}

% TODO: REVISAR
Nowadays it's impossible to think in live without digital technologies. They
are involved in most of our daily actions, such as, to talk with friends, to
read news and entertainment. For this reason, our electronic devices can be
considered as digital sources of information that can be used to prove even
what it's owner has done at a specific moment.

Digital forensics science, or digital forensics, is defined as:

\begin{quote}
"The use of scientifically derived and proven methods toward the preservation,
collection, validation, identification, analysis, interpretation,
documentation, and presentation of digital evidence derived from digital
sources for the purpose of facilitation or furthering the reconstruction of
events found to be criminal, or helping to anticipate unauthorized actions
shown to be disruptive to planned operations" \cite{DFRWS-df-road-map}
\end{quote}

In several cases, when a search warrant is given to investigators, they can
proceed with the acquisition and analysis of a suspect's digital data source,
for instance, hard disks (HDD). Using this information, they can look for
evidences that prove the suspect's innocence or guiltiness.

There are many ways to do this analysis but the more generic one presented in
\cite{ds-phases} consists in:

\begin{description}
	\item [Pre-Process]
		Even sometimes this step could be forgotten, is one of the more
		important. It consists in all the works needed to be done before
		the collection of data. To have all the approvals for relevant
		authorities is considered here.

	\item [Acquisition \& Preservation]
		Once everything is prepared, now investigator can acquire the
		data. An image and a hash of the image have to be computed. The
		image is an exact copy of the data and the hash is a small piece
		of information that summarizes the image.

		That way, if anyone modifies the image and the hash is computed
		again, a different value will be retrieved. So this procedure
		stands to guarantee the proof integrity.

	\item [Analysis]
		Is the main phase of the investigation. It consists on extract
		the relevant information acquired before. Many different 
		techniques can be used: data recovery, words lookup, etc. Note
		that, on this step, many different tools may be used to be able
		to perform a proper analysis.

	\item [Presentation]
		A good analysis is useless if it's not presented in a easy to 
		understand manner. After this phase, the innocence of the
		suspect has to be proved or refused.

	\item [Post-Process]
		Physical (i.e. HDD) and digital proofs have to be returned and
		a document must be written to finish the case.
\end{description}

There are many digital forensics analysis tools but most of them are not
supported on all the operative systems. A very complete software is 
\textbf{EnCase Forensic} \cite{encase-web}. It has a lot of tools that can be 
used from the acquisition step until the final report. But it's main 
disadvantage is that is not open source.

Another well known software regarding digital forensics is \textbf{The Sleuth
Kit} \cite{tsk-web}. This open source cross-platform toolkit is a collection 
of terminal applications and a C library that allows you to analyze disk images 
and recover files from them. The Sleuth Kit team also developed
\textbf{Autopsy}, with is a windows user interface to efficiently analyze hard 
drives and smart phones.

Due to the lack of open source cross-platform tools with user interface, the
objective of this project develop one with main functionalities. The start point
is to use \textbf{The Sleuth Kit}'s C library since it's cross-platform and a 
very mature project.

