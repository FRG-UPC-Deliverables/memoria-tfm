\chapter{Compaginació}\label{C:compaginacio}

\section{Paper i impressió}

\subsection{Paper}

Cal utilitzar paper mida DIN A4 vertical (210 x 297 mm)  que és l'estàndard més generalitzat.

\subsection{Impressió a dues cares}

La presentació del document ha de ser a dues cares a partir de la Introducció i fins al final del document.


\section{Marges}

Pel que fa als marges, no cal fer absolutament res. Aquesta plantilla \LaTeX \ ja ho fa per vosaltres :-)


\section{Tipografia}

\subsection{Tipus de lletra}

Pel que fa al tipus de lletra, no cal fer absolutament res. Aquesta plantilla \LaTeX \ ja ho fa per vosaltres :-)

A part dels capítols, només podeu utilitzar tres nivells de profunditat en la divisió per apartats:

\begin{verbatim}
\chapter{Nom del capítol}
\section{Nom de l'apartat}
\subsection{Nom del sub-apartat}
\subsubsection{Nom del sub-sub-apartat}
\end{verbatim}



\subsection{Interlineat}

Pel que fa al interlineat, ja sigui entre línies o finals de paràgrafs/seccions/capitols, no cal fer absolutament res. Aquesta plantilla \LaTeX \ ja ho fa per vosaltres :-)


\section{Numeració dels títols}

Pel que fa a la numeració dels capítols, apartats, subapartats i subsubapartats, no cal fer absolutament res. Aquesta plantilla \LaTeX \ ja ho fa per vosaltres :-)


\section{Encapçalaments i números de pàgina}

Pel que fa als encapçalaments i números de pàgina, no cal fer absolutament res. Aquesta plantilla \LaTeX \ ja ho fa per vosaltres :-)


\section{Enquadernació}

Cal realitzar l'enquadernació amb espiral negra, i tapes de plàstic transparent. La portada ha de ser de cartolina de color blau (color Pantone 542 o el més similar possible).
