\chapter{Digital forensics analysis}
\label{S:df-analysis}

The Sleuth Kit (TSK) is an open source cross-platform collection of terminal
applications and a C library that allows investigators to analyze disk images
and recover files from them \cite{tsk-web}. It is divided in file system and
volume tools. 

The first group allows investigators to examine the file system of a suspect
computer in a non-intrusive way. Main file system tools are shown on Table
\ref{T:tsk-fs-tools}.

\begin{table}[htb]
\begin{center}
\begin{tabular}{|l|l|l|}
\hline
{\bf Command }	& {\bf Description }  \\ \hline \hline
fsstat & Shows file system details and statistics \\ \hline \hline
fls	& Lists allocated and deleted file names in a directory \\ \hline
ffind & Finds allocated and unallocated file names \\ \hline \hline
icat & Extracts the data units of a file \\ \hline
\end{tabular}
\caption{Main file system tools from TSK \cite{tsk-tools-wiki}}
\label{T:tsk-fs-tools}
\end{center}
\end{table}

Volume tools are the ones that brings off the layout of disks and another media.
They are needed to identify where partitions are located to be able to proceed
with a file system analysis. Table \ref{T:tsk-v-tools} lists main volume tools.

\begin{table}[htb]
\begin{center}
\begin{tabular}{|l|l|l|}
\hline
{\bf Command }	& {\bf Description }  \\ \hline \hline
mmls & Displays the layout of a disk \\ \hline \hline
mmstat & Display details about a volume system \\ \hline
\end{tabular}
\caption{Main volume tools from TSK \cite{tsk-tools-wiki}}
\label{T:tsk-v-tools}
\end{center}
\end{table}

In order to explain how The Sleuth Kit works easily, an example will be used.
In that specific case, an investigator will extract the timeline for an image
and will look for a file called \textit{letter.txt} to print its content.

\section{The Sleuth Kit analysis}
\label{S:tsk-analysis}

To start the TSK analysis, the command \textit{mmls} is used to output the 
information of all partitions. If volume type is not specified, TSK will
autodetect it. The output will look like Listing \ref{L:tsk-mmls-output}.
This example has tree volumes:

\begin{description}
	\item [Primary Table (\#0)]
	It contains the primary table. 
	
	\item [Unallocated]
	Unallocated space.

	\item [Win95 FAT32 (0x0C)]
	This partition has an offset of 56 sectors and it contains a file system formatted with FAT32.
	
\end{description}

\begin{terminal}[caption=TSK mmls output example, label=L:tsk-mmls-output]
	%
	\terminalcmd[mmls hdd-001.dd]
	%
	DOS Partition Table
	Offset Sector: 0
	Units are in 512-byte sectors
	
	Slot         Start        End          Length       Description
	00:  Meta    0000000000   0000000000   0000000001   Primary Table (#0)
	01:  -----   0000000000   0000000055   0000000056   Unallocated
	02:  00:00   0000000056   0007925759   0007925704   Win95 FAT32 (0x0C)
	%\terminalcmd%
\end{terminal}

To perform a deeper analysis, file system tools must be used. In that case,
those will need the offset of 56 sectors to be specified. The  first tool,
\textit{fsstat}, will display some file system characteristics, for instance,
root inode, sector size and the file system layout.

Also, the registers inside the disk can be listed in a non-intrusive way using 
\textit{fls}. In this specific example (Listing \ref{L:fls-output-example}),
the file system contains four folders. One of them,  
\textit{\_NTITL\textasciitilde1} is deleted. Also, some virtual registers are
displayed (\$MBR, \$FAT1, \$FAT2, \$OrphanFiles). If it is necessary to go
through the file system tree, a directory inode can be specified to list
registers inside it or analyze whole disk recursively.

\begin{terminal}[caption=TSK fls output example,label=L:fls-output-example]
%
\terminalcmd[fls -o 56 hdd-001.dd]
%
d/d 4:	.Trash-1000
d/d * 5:	_NTITL~1
d/d 7:	Personal
d/d 9:	Imagenes
v/v 126563459:	$MBR
v/v 126563460:	$FAT1
v/v 126563461:	$FAT2
d/d 126563462:	$OrphanFiles
%\terminalcmd%
\end{terminal}

Several times, it is necessary to export the whole file system in order to obtain
processed information using other tools. The command arguments that are
necessary are:

\begin{terminal}[caption=Export file system to CSV,label=L:fls-csv-export]
%
\terminalcmd[fls -m / -o 56 -r hdd-001.dd > fs.csv]
%
%\terminalcmd%

\end{terminal}

This CSV file can also be used to generate timelines using a specific TSK tool 
called \textit{mactime}. The output format of \textit{mactime} can be also
adjusted to process it with any CSV program. Columns shown on Listing
\ref{L:tsk-timeline} are: date, file size, activity type, Unix permission, User
\& Group IDs, inode and file name. Allowed activity types are: modified (m),
accessed (a), changed (c) and created (b).

\begin{terminal}[caption=Haxe python transcompilation command, label=L:tsk-timeline]
%
\terminalcmd[haxe -main Main.hx -python main.py]
%
Xxx Xxx 00 0000 00:00:00   338558 ..c. r/rrwxrwxrwx 0        0        10758    /Imagenes/bufon.png
         10 ..c. r/rrwxrwxrwx 0        0        10889    /Personal/file.txt
       4096 ..c. d/drwxrwxrwx 0        0        11142    /.Trash-1000/info
       4096 ..c. d/drwxrwxrwx 0        0        11144    /.Trash-1000/files
         70 ..c. r/rrwxrwxrwx 0        0        11274    /.Trash-1000/info/carta.txt.trashinfo
         72 ..c. r/rrwxrwxrwx 0        0        11401    /.Trash-1000/files/carta.txt
       4096 ..c. d/drwxrwxrwx 0        0        4        /.Trash-1000
          0 ..c. d/drwxrwxrwx 0        0        5        /_NTITL~1 (deleted)
       4096 ..c. d/drwxrwxrwx 0        0        7        /Personal
       4096 ..c. d/drwxrwxrwx 0        0        9        /Imagenes
Fri Aug 25 2017 00:00:00   338558 .a.. r/rrwxrwxrwx 0        0        10758    /Imagenes/bufon.png
         10 .a.. r/rrwxrwxrwx 0        0        10889    /Personal/file.txt
       4096 .a.. d/drwxrwxrwx 0        0        11142    /.Trash-1000/info
       4096 .a.. d/drwxrwxrwx 0        0        11144    /.Trash-1000/files
         70 .a.. r/rrwxrwxrwx 0        0        11274    /.Trash-1000/info/letter.txt.trashinfo
         72 .a.. r/rrwxrwxrwx 0        0        11401    /.Trash-1000/files/letter.txt
       4096 .a.. d/drwxrwxrwx 0        0        4        /.Trash-1000
          0 .a.. d/drwxrwxrwx 0        0        5        /_NTITL~1 (deleted)
       4096 .a.. d/drwxrwxrwx 0        0        7        /Personal
       4096 .a.. d/drwxrwxrwx 0        0        9        /Imagenes
Fri Aug 25 2017 15:06:56   338558 m..b r/rrwxrwxrwx 0        0        10758    /Imagenes/bufon.png
Fri Aug 25 2017 15:07:18     4096 m..b d/drwxrwxrwx 0        0        9        /Imagenes
Fri Aug 25 2017 15:07:22        0 m..b d/drwxrwxrwx 0        0        5        /_NTITL~1 (deleted)
Fri Aug 25 2017 15:07:30     4096 m..b d/drwxrwxrwx 0        0        4        /.Trash-1000
Fri Aug 25 2017 15:08:04       72 m..b r/rrwxrwxrwx 0        0        11401    /.Trash-1000/files/carta.txt
Fri Aug 25 2017 15:08:06       70 m..b r/rrwxrwxrwx 0        0        11274    /.Trash-1000/info/carta.txt.trashinfo
Fri Aug 25 2017 15:08:12     4096 m..b d/drwxrwxrwx 0        0        11142    /.Trash-1000/info
< 4096 m..b d/drwxrwxrwx 0        0        11144    /.Trash-1000/files
Fri Aug 25 2017 15:08:22       10 m..b r/rrwxrwxrwx 0        0        10889    /Personal/file.txt
       4096 m..b d/drwxrwxrwx 0        0        7        /Personal
%\terminalcmd%

\end{terminal}

As Listing \ref{L:tsk-timeline} shows, the inode 11401 contains a file called
letter.txt. Print its content can be easily done using the command
\textit{icat}. 


\begin{terminal}[caption=Export a file contained inside the file system,label=L:icat-export]
%
\terminalcmd[icat -o 56 hdd-001.dd 11401]
%
Mr. Someone,
You must know I am guilty!

Best regards,
Nobody Jones
%\terminalcmd%

\end{terminal}


\section{The Sleuth Kit JavaScript}

% TODO:
% Quiero dar enfasis en que esto es parte del desarrollo del proyecto y no se 
% si esta bien hecho asi...

Bearing in mind the need to develop a wrapper to connect TSK with a JavaScript
application \textbf{The Sleuth Kit JavaScript (TSK-js) has been created}. This
wrapper must be able to execute the same analysis carried out in the previous chapter
(Chapter \ref{S:tsk-analysis}) but using JavaScript.

It has been raised to create a JavaScript object with one construction argument
in with the image file path is specified. This object will have the necessary
methods to perform the entire analysis. Those methods are explained on Table
\ref{T:tsk-js-object-methods}.

\begin{table}[htb]
\begin{center}
\begin{tabular}{|l|l|l|}
\hline
{\bf Method }	& {\bf Arguments} & {\bf Returns } \\ \hline \hline
Analyze	& None & Type (disk/partition) and if type disk, returns\\
& 	   & its partitions specifying the offset sectors \\ \hline
List & Offset sectors and inode & List file system files \\ \hline
Get & Offset sectors and inode & Buffer with inode content \\ \hline
Timeline & Offset sectors, inode & Timeline results \\ 
& and callback function & \\ \hline
Search & Needle, offset sectors, & None \\ 
& inode and callback function & \\ \hline
\end{tabular}
\caption{TSK-js object methods}
\label{T:tsk-js-object-methods}
\end{center}
\end{table}

Search and Timeline take some time to execute. That is why both have a callback
function as parameter. Each time a result is found, this callback function is
called.

An example of JavaScript code that can perform the analysis proposed on 
Chapter \ref{S:tsk-analysis} is shown of Listing \ref{L:tsk-js-analysis}.

\begin{codefigure}
	\jsexternal[
		caption=The Sleuth Kit JavaScript analysis,
		label=L:tsk-js-analysis,
		%
		classoffset=1,
		morekeywords={TSK},
	]{source/tsk-js-analysis.js}
\end{codefigure}

This analysis can be executed using the following command:

\begin{terminal}[caption=Execute The Sleuth Kit JavaScript analysis
,label=L:tsk-js-execute-analysis]
%
\terminalcmd[node analysis.js hdd-001.dd]
%
File found!
Print it's content:
---------------------------
Mr. Someone,
You must know I am guilty!

Best regards,
Nobody Jones
---------------------------
Timeline length '29'
%\terminalcmd%

\end{terminal}

The implementation of this package is done using \CC and node-gyp that is a 
compiler to create Node.js addons \cite{node-gyp-github}.

% TODO: Explicar compilacion y project structure.

A TypeScript types file is provided in order to help transcompiler with type 
validations. Typings can be found on Appendix \ref{APP:tsk-js-typings}.
This package is available on Node Package Manager
(\url{https://www.npmjs.com/package/tsk-js}). The source code is also available
on GitHub (\url{https://github.com/FRG-UPC-Deliverables/tsk-js}).

