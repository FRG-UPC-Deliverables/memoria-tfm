\chapter{Cross-platform development}

\cite{soft-that-lasts}

At the beginning of computers, each hardware manufacturer had to decide how the
programmer talks with its hardware. This fact created the cross-platform
compatibility problem.

When Operative Systems (OS) appeared, this problem was reduced because at that
moment there is no need to create a different program for each manufacturer,
just for each operative system. This is due to the operative systems have
different drivers, provided for manufacturers, that logical functions to
hardware operations.

But the problem is not fully solved since the time to develop an application
for several OS is to expensive in most cases.

\section{Technological soup}

Nowadays lots of technologies has been created with the goal of writing just
one code to run in all of the current platforms.

\subsection{Java}

\subsection{Qt}

\subsection{Python}

Using Kivy, a python GUI framework.

\subsection{Electron}

\subsection{NW.js}

\subsection{Haxe}

Native programming language

\section{Electron}

\section{Web technologies}

\subsection{Typescript}

\subsection{Sass}

\subsection{Gulp}

