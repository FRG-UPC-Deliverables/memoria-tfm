\chapter{Cross-platform development}

At the beginning of computers, each hardware manufacturer had to decide how to
access to its hardware functionalities. Since not all of them were decided to be
the same the fist inter-compatibility problems started to appear because 
programs were able to run on different computers.

To reduce this problems drivers were created. A driver is a piece of code that 
maps logical functions (for instance, turn on a led) to hardware operations 
(short circuit some part of the hardware). Logic functions are provided to 
programs by the Operative Systems (OS) so the problem now is that not all of 
them can run on all the OS.

\section{Technological soup}

Nowadays lots of technologies has been created with the goal of writing just
one code to run in many platforms as possible. This reduces a lot the
development cost by reducing performance or other factors .

The most common way to achieve this goal is with scripting or virtual machine 
programming languages. Those languages are interpreted on execution time. 
Therefore programs created with this kind of technologies can run on all
platforms supported by the interpreter.

Another kind of solutions are transpilers or source-to-source compilers. They 
translate from one programming language to many others. This languages are
able to be executed on many platforms.

\subsection{Java}

This well known programming language is one of the most used to build
cross-platform applications due to its maturity. 

\subsection{Qt}

\subsection{Python}

Using Kivy, a python GUI framework.

\subsection{Electron}

\subsection{NW.js}

\subsection{Haxe}

Native programming language

\section{Electron}

\section{Web technologies}

\subsection{Typescript}

\subsection{Sass}

\subsection{Gulp}

