\chapter{Cross-platform development}

At the beginning of computers, each hardware manufacturer had to decide how to
access to its hardware functionalities. Since there where no standards for some
cases, the first inter-compatibility problems started to appear because
programs were not able to run on different computers.

To reduce this problems drivers were created. A driver is a piece of code that 
maps logic functions (for instance, turn on a led) to hardware operations 
(short circuit some part of the electronics). Programs call this logic functions
so for them it's not important how to turn on the led, is the driver who 
specifies how to do it. Then for software, the only important part is how this
functions are called.

Logic functions are provided to programs by the Operative Systems (OS) so the 
problem now is that not all OS use the same name to this functions.

\section{Technological soup}

Nowadays lots of technologies has been created with the goal of writing just
one code to run in as many platforms as possible. We want to achieve this
to reduce the development cost but this results in a performance reduction.

An intuitive solution may be to use cross-platform frameworks to perform the 
functions that have problems with interoperability. This frameworks have 
conditionals that execute a logic function or another depending on which 
operative system is the one that is compiling this library.

But the most common way to achieve this goal is with scripting programming
languages. This languages are human readable (text) that are interpreted by a
program called interpreter. One step further is to compile this human readable
text to pseudo-machine code that easy to understand for the interpreter, also
known as virtual machine programming languages. This have the advantage that 
programs are faster, but developers lose some time compiling the code each time 
they want to test something. Both technologies can run over all the platforms
that its interpreter is supported.

Another kind of solutions are transcompilers or source-to-source compilers. They 
translate from one programming language to many others. This translation could 
be done in a scripting language (there interoperability is provided) or can make
an specific translation to a compiled language depending on the OS.

\subsection{Java}

This well known programming language is one of the most used to build
cross-platform applications due to its maturity. Java is a general-purpose,
concurrent, classbased, object-oriented language and it's designed to be easy
to learn in order to let many programmers to have fluency in the language
\cite{java-8-specs}.

Classic compiled languages, such as, C and C++, directly compile into machine
code, which is directly interpreted by the hardware. As opposed to C and C++,
Java can be considered a virtual machine programming language.

\begin{codefigure}{Java hello world}{F:java-hello-world}
	\javaexternal{source/HelloWorldApp.java}
\end{codefigure}

Figure \ref{F:java-hello-world} is an example of Java program that prints the
test "Hello World!" to the terminal.

\subsection{Qt}

A complete cross-platform software framework with ready-made UI elements, C++ 
libraries, and a complete integrated development environment with tools for 
everything you need to develop software for any project \cite{qt-web}.

\begin{codefigure}{Qt hello world}{F:qt-hello-world}
	\cppexternal{source/qt_hello_world.cpp}
\end{codefigure}

\subsection{Python}

This is an example of an scripting programming language. The main design rule
was to create a language easy to read. That's why Python is very famous due to
its strong tabulation rules.

Its also a considered an object-oriented language suitable for many purposes.
It has a clear, intuitive syntax, powerful high-level data structures, and a
flexible dynamic type system \cite{An93pythonfor}.

Python is often used to create terminal programs or web services because GUI
are not supported by default but there are many frameworks that can be used,
for instance, \textbf{Kivy}.

\begin{codefigure}{Kivy hello world}{F:kivy-hello-world}
	\pythonexternal{source/kivy_hello_world.py}
\end{codefigure}

\subsection{Electron}

\subsection{NW.js}

\subsection{Haxe}

Native programming language

\section{Electron}

\section{Web technologies}

\subsection{Typescript}

\subsection{Sass}

\subsection{Gulp}

