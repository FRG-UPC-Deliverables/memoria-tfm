\chapter{Digital forensics}

% TODO
De wikipedia! Redatar otra vez

Digital forensics (sometimes known as digital forensic science) is a branch of forensic 
science encompassing the recovery and investigation of material found in digital devices,
often in relation to computer crime.[1][2] The term digital forensics was originally used
is a synonym for computer forensics but has expanded to cover investigation of all devices 
capable of storing digital data.[1] With roots in the personal computing revolution of
the late 1970s and early 1980s, the discipline evolved in a haphazard manner during the 
1990s, and it was not until the early 21st century that national policies emerged.

The goal of computer forensics is to explain the current state of a digital artifact; such
as a computer system, storage medium or electronic document.[38] The discipline usually
covers computers, embedded systems (digital devices with rudimentary computing power and
onboard memory) and static memory (such as USB pen drives).

Computer forensics can deal with a broad range of information; from logs (such as internet
history) through to the actual files on the drive.

In 2007 prosecutors used a spreadsheet recovered from the computer of Joseph E. Duncan III
to show premeditation and secure the death penalty.[3] Sharon Lopatka's killer was
identified in 2006 after email messages from him detailing torture and death fantasies were
found on her computer.

\section{Phases}

% TODO
SOME TEXT HERE

\begin{description}
	\item [Attribution]
		Meta data and other logs can be used to attribute actions to an individual. 
		
		For example, personal documents on a computer drive might identify its owner.
	\item [Alibis and statements]
		Information provided by those involved can be cross checked with digital 
		evidence.
		
		For example, during the investigation into the Soham murders the offender's
		alibi was disproved when mobile phone records of the person he 	claimed to be
		with showed she was out of town at the time.
	\item [Intent]
		As well as finding objective evidence of a crime being committed, investigations
		can also be used to prove the intent (known by the legal term mens rea).
		
		For example, the Internet history of convicted killer Neil Entwistle included 
		references to a site discussing How to kill people.
	\item [Evaluation of source]
		File artifacts and meta-data can be used to identify the origin of a particular
		piece of data; for example, older versions of Microsoft Word embedded a Global
		Unique Identifer into files which identified the computer it had been created on. 
		Proving whether a file was produced on the digital device being examined or
		obtained from elsewhere (e.g., the Internet) can be very important.[3]
	\item [Document authentication]
		Related to "Evaluation of source," meta data associated with digital documents
		can be easily modified (for example, by changing the computer clock you can affect
		the creation date of a file). Document authentication relates to detecting and
		identifying falsification of such details.
\end{description}

\section{Tools}

% TODO
What's a digital forensic tool??

\subsection{The Sleuth Kit and Autopsy}


