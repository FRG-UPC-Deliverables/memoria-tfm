\chapter{Digital forensics}

% TODO
De wikipedia! Redatar otra vez

Digital forensics (sometimes known as digital forensic science) is a branch of
forensic science encompassing the recovery and investigation of material found
in digital devices, often in relation to computer crime.[1][2] The term digital
forensics was originally used is a synonym for computer forensics but has
expanded to cover investigation of all devices capable of storing digital
data.[1] With roots in the personal computing revolution of the late 1970s and
early 1980s, the discipline evolved in a haphazard manner during the 1990s, and
it was not until the early 21st century that national policies emerged.

The goal of computer forensics is to explain the current state of a digital
artifact; such as a computer system, storage medium or electronic document.[38]
The discipline usually covers computers, embedded systems (digital devices with
rudimentary computing power and onboard memory) and static memory (such as USB
pen drives).

Computer forensics can deal with a broad range of information; from logs (such
as internet history) through to the actual files on the drive.

In 2007 prosecutors used a spreadsheet recovered from the computer of Joseph E.
Duncan III to show premeditation and secure the death penalty.[3] Sharon
Lopatka's killer was identified in 2006 after email messages from him detailing
torture and death fantasies were found on her computer.

\section{Phases}

% TODO
SOME TEXT HERE



\section{Tools}

% TODO
What's a digital forensic tool??

\subsection{The Sleuth Kit and Autopsy}

\subsection{The Coroner's Toolkit}

